\documentclass{econsocart}
\usepackage{../@local/local-qe-figs-and-tables}

\begin{document}

% Calibrated Model Parameters Table - Recession
\begin{table}[tb] 
  \caption{Calibrated Model Parameters --- Recession}
  \label{tab:calibrationRecession} 
  \centering

  % Ensure tables use full text width without minipage constraints

  % Panel A - Parameters that apply to all types that change in recession
  \begin{tabular*}
    {\textwidth}{@{\extracolsep{\fill}}lcr@{}}
    \multicolumn{3}{c}{\small Panel A: Parameters that apply to all types} \\
    \addlinespace
    \hline
    Parameter                                           & Notation    & \text{Value} \\ \hline
    Avg.\ duration of unemp.\ spell in a recession (quarters) &         & 4            \\
    Prob.  of leaving unemployment in a recession  & $\pi_{ue}$ & 0.25         \\
    \hline
    \multicolumn{3}{l}{\textcolor{white}{.}} \\  % Invisible spacing
  \end{tabular*}

  \medskip

  % Panel B - Education-specific parameters that change in recession
  \begin{tabular*}
    {\textwidth}{@{\extracolsep{\fill}}lccc@{}}
    \multicolumn{4}{c}{\small Panel B: Parameters calibrated for each education group} \\
    \addlinespace
    \hline
    & Dropout      & Highschool & College \\ \hline
    Unemp.\ rate at the start of a recession (percent) & \phantom{0}17.0 & \phantom{0}8.8 & \phantom{0}5.4 \\
    Prob.\ of entering unemployment ($\pi_{eu}^{e}$, percent) & \phantom{0}5.1 & \phantom{0}2.4 & \phantom{0}1.4 \\
    \hline
    \multicolumn{4}{l}{\textcolor{white}{.}} \\  % Invisible spacing
  \end{tabular*}

  \medskip

  % Panel C - Parameters describing policy experiments
  \begin{tabular*}
    {\textwidth}{@{\extracolsep{\fill}}lr@{}}
    \multicolumn{2}{c}{\small Panel C: Parameters describing policy experiments} \\
    \addlinespace
    \hline
    Parameter                                        & Value \\ \hline
    Average length of recession                      & 6 quarters \\
    Size of stimulus check                           & \$1,200 \\
    PI threshold for reducing check size             & \$100,000 \\
    PI threshold for not receiving check             & \$150,000 \\
    Extended unemployment benefits                   & 4 quarters \\
    Length of payroll tax cut                        & 8 quarters \\
    Income increase from payroll tax cut             & 2 percent \\
    Belief (probability) that tax cut is extended    & 50 percent \\
    \hline
  \end{tabular*}

  \vspace{0.5em}
  \noindent\parbox{\textwidth}{\footnotesize
    \textbf{Note}: ``PI'' refers to Permanent Income}
  \vspace{0.5em}
\end{table}


\end{document}
