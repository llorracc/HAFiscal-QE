\documentclass{econsocart}
\usepackage{../@local/local-qe-figs-and-tables}

\begin{document}

% Welfare comparison across policies with and without splurge
\begin{table}[tb] 
  \caption{Welfare for policies both out of and in a recession, with/without AD effects}
  \label{tab:welfare6-SplurgeComp} 
  \centering

  \begin{tabular*}
    {\textwidth}{@{\extracolsep{\fill}}lrrr@{}} 
    \hline
    & \multicolumn{1}{c}{Stimulus check} & \multicolumn{1}{c}{UI extension} & \multicolumn{1}{c}{Tax cut} \\ \hline
    $\mathcal{W}(\text{policy}, \texttt{Rec=0}, \texttt{AD=0})$ & 0.97(0.96) & 0.84(0.85) & 0.99(0.99) \\
    \addlinespace
    $\mathcal{W}(\text{policy}, \texttt{Rec=1}, \texttt{AD=0})$ & 1.00(1.00) & 1.80(1.83) & 0.97(0.97) \\
    $\mathcal{W}(\text{policy}, \texttt{Rec=1}, \texttt{AD=1})$ & 1.27(1.35) & 2.12(2.15) & 1.09(1.11) \\
    \hline
  \end{tabular*}

  % Table note
  \noindent\parbox{\textwidth}{
    \medskip
    \footnotesize Note: The values outside of the brackets capture the welfare in the model without the splurge, while those inside the brackets are welfare with the splurge. \texttt{Rec=0} indicates normal times, \texttt{Rec=1} indicates recession. \texttt{AD=0/1} indicates aggregate demand effects inactive/active.
  }
\end{table}

\vspace{0.5em}

\end{document}
