\documentclass{econsocart}
\input{./._relpath-to-latexroot.ltx}
\usepackage{../@local/local-qe-figs-and-tables}

\begin{document}

% Policy multiplier comparison with and without splurge
\begin{table}[tb] 
  \caption{\pregenmark Policy multipliers w.\ and w/o.\ splurge: recession implementation}
  \label{tab:Multiplier-SplurgeComp} % integrated doc: SST for labels
  \centering

  \begin{tabular*}
    {\textwidth}{@{\extracolsep{\fill}}lccc@{}}
    \hline
    & Stimulus check & UI extension & Tax cut \\ \hline
    10y-horizon Multiplier (no AD effect) & 0.870 (0.879) & 0.910 (0.906) & 0.839 (0.847) \\
    10y-horizon Multiplier (AD effect)    & 1.143 (1.234) & 1.221 (1.211) & 0.947 (0.978) \\
    \hline
  \end{tabular*}

  % Table note
  \noindent\parbox{\textwidth}{
    \medskip
    \footnotesize Note: Values outside brackets show multipliers in the model without the splurge, while those inside brackets are the corresponding multipliers with the splurge. Policies are implemented during a recession with or without aggregate demand (AD) effects active. Despite substantial MPC differences across model variants (see Table \ref{tab:Comparison-Splurge-Table}), multiplier shifts are only minor, though the lower average MPCs reduce effectiveness of check and tax cut policies relative to UI extension.
  }
\end{table}

\vspace{0.5em}


\end{document}
