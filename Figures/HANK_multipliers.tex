\documentclass{econsocart}
\usepackage{../@local/local-qe-figs-and-tables}

\begin{document}

\begin{figure}[htb] 
  \centering
  \caption{Multiplier comparison: partial equilibrium vs.\ HANK model}
  \label{fig:HANK_multipliers} 
  % Original path: \latexroot/Code/HA-Models/FromPandemicCode/Figures/Cumulative_multipliers_withHank
  \includegraphics[width=.9\textwidth]{\latexroot/images/Cumulative_multipliers_withHank}

  \medskip
  \noindent\parbox{\textwidth}{\footnotesize
    \textbf{Note}: This figure compares consumption multipliers across model frameworks (Section~\ref{sec:hank}).
    The comparison shows multipliers over different time horizons under a fixed real rate rule
    between the baseline partial equilibrium model (with state-dependent multipliers during recessions)
    and the general equilibrium HANK-SAM model (with uniform Keynesian cross mechanism).
    Both models show the tax cut policy substantially underperforming relative to stimulus checks
    and UI extensions, validating the robustness of policy rankings across modeling approaches.
    The HANK model produces larger overall multipliers due to general equilibrium effects.
  }
\end{figure}

\vspace{0.5em}


\end{document}
